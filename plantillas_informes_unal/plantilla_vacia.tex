% Informe - Universidad Nacional de Colombia
% Plantilla vacía lista para usar

\documentclass[
	spanish,
	letterpaper, oneside
]{article}

% ==================== INFORMACIÓN DEL DOCUMENTO ====================
% MODIFICAR ESTOS VALORES SEGÚN TU TAREA/PROYECTO

\def\documenttitle {Título del Trabajo}
\def\documentsubtitle {Subtítulo o Número de Tarea}
\def\documentsubject {Tema General del Documento}

\def\documentauthor {Nombre Estudiante 1, Nombre Estudiante 2}
\def\coursename {Nombre del Curso}
\def\coursecode {Código - Semestre}

\def\universityname {Universidad Nacional de Colombia}
\def\universityfaculty {Facultad de Ciencias}
\def\universitydepartment {Departamento de Matemáticas}
\def\universitydepartmentimage {departamentos/unal}
\def\universitydepartmentimagecfg {height=1.57cm}
\def\universitylocation {Bogotá D.C., Colombia}

% ==================== INTEGRANTES Y FECHAS ====================
\def\authortable {
	\begin{tabular}{ll}
		Estudiantes:
		& \begin{tabular}[t]{l}
			Nombre Completo Estudiante 1 \\
			correo1@unal.edu.co \\
			\\
			Nombre Completo Estudiante 2 \\
			correo2@unal.edu.co
		\end{tabular} \\
		& \\
		Profesor:
		& \begin{tabular}[t]{l}
			Nombre del Profesor
		\end{tabular} \\
		& \\
		\multicolumn{2}{l}{Fecha de entrega: DD de mes de YYYY} \\
		\multicolumn{2}{l}{\universitylocation}
	\end{tabular}
}

% ==================== IMPORTACIÓN DEL TEMPLATE ====================
\input{plantilla_src/template}

% ==================== INICIO DE PÁGINAS ====================
\begin{document}

% PORTADA
\templatePortrait

% CONFIGURACIÓN DE PÁGINA Y ENCABEZADOS
\templatePagecfg

% RESUMEN/ABSTRACT
\begin{abstractd}
Escribe aquí el resumen de tu trabajo. Este debe ser un párrafo conciso que describa el objetivo, metodología y principales resultados del documento. Generalmente entre 150-250 palabras.
\end{abstractd}

% TABLA DE CONTENIDOS
\templateIndex

% CONFIGURACIONES FINALES
\templateFinalcfg

% ==================== CONTENIDO DEL DOCUMENTO ====================

\section{Introducción}

Escribe aquí la introducción de tu documento. La introducción debe contextualizar el problema, presentar los objetivos y describir la estructura del documento.

\subsection{Contexto}

Desarrollo del contexto...

\subsection{Objetivos}

\begin{itemize}
    \item Objetivo 1
    \item Objetivo 2
    \item Objetivo 3
\end{itemize}

\section{Marco Teórico}

Presenta aquí los conceptos teóricos necesarios para entender tu trabajo.

\subsection{Conceptos Fundamentales}

Desarrollo de conceptos...

\begin{defn}[Nombre de la Definición]
Enunciado de la definición...
\end{defn}

\begin{teo}[Nombre del Teorema]
Enunciado del teorema...
\end{teo}

\section{Metodología}

Describe la metodología utilizada en tu trabajo.

\subsection{Implementación}

Detalles de implementación...

\section{Resultados}

Presenta los resultados obtenidos.

\subsection{Análisis de Resultados}

Análisis detallado...

% Ejemplo de tabla
\begin{table}[htbp]
\centering
\caption{Título de la tabla}
\label{tab:ejemplo}
\begin{tabular}{|c|c|c|}
\hline
Columna 1 & Columna 2 & Columna 3 \\
\hline
Dato 1 & Dato 2 & Dato 3 \\
Dato 4 & Dato 5 & Dato 6 \\
\hline
\end{tabular}
\end{table}

% Ejemplo de figura
% \insertimage[\label{fig:ejemplo}]{ruta/imagen.png}{width=12cm}{Descripción de la figura}

\section{Conclusiones}

Resume las conclusiones principales del trabajo.

\begin{enumerate}
    \item Primera conclusión
    \item Segunda conclusión
    \item Tercera conclusión
\end{enumerate}

\subsection{Trabajo Futuro}

Posibles extensiones del trabajo...

% ==================== REFERENCIAS ====================
\section{Referencias}

\begin{thebibliography}{9}
\bibitem{ref1}
Autor, A. (Año). \textit{Título del libro o artículo}. Editorial o Journal.

\bibitem{ref2}
Autor, B., Autor, C. (Año). Título del artículo. \textit{Nombre de la Revista}, Volumen(Número), páginas.

\end{thebibliography}

% ==================== APÉNDICES (OPCIONAL) ====================
% \newpage
% \appendix
%
% \section{Código Fuente}
% Código adicional...
%
% \section{Resultados Adicionales}
% Datos complementarios...

% ==================== FIN DEL DOCUMENTO ====================
\end{document}
