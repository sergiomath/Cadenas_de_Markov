% Informe Tarea 1 - Cadenas de Markov y Aplicaciones

\documentclass[
	spanish,
	letterpaper, oneside
]{article}

% INFORMACIÓN DEL DOCUMENTO
\def\documenttitle {Aplicación del muestreo de Gibbs a los modelos Hard-Core y q-coloraciones}
\def\documentsubtitle {Tarea 1}
\def\documentsubject {Muestreo de Gibbs en Modelos Estocásticos en Rejillas}

\def\documentauthor {Sergio Andrés Díaz Vera, Julián Mateo Espinosa Ospina}
\def\coursename {Cadenas de Markov y Aplicaciones}
\def\coursecode {2025-II}

\def\universityname {Universidad Nacional de Colombia}
\def\universityfaculty {Facultad de Ciencias}
\def\universitydepartment {Departamento de Matemáticas}
\def\universitydepartmentimage {departamentos/unal}
\def\universitydepartmentimagecfg {height=1.57cm}
\def\universitylocation {Bogotá D.C., Colombia}

% INTEGRANTES, PROFESORES Y FECHAS
\def\authortable {
	\begin{tabular}{ll}
		Estudiantes:
		& \begin{tabular}[t]{l}
			Sergio Andrés Díaz Vera \\
			seadiazve@unal.edu.co \\
			\\
			Julián Mateo Espinosa Ospina \\
			juespinosao@unal.edu.co
		\end{tabular} \\
		& \\
		Profesor:
		& \begin{tabular}[t]{l}
			Freddy Hernández-Romero
		\end{tabular} \\
		& \\
		\multicolumn{2}{l}{Fecha de entrega: 30 de octubre de 2025} \\
		\multicolumn{2}{l}{\universitylocation}
	\end{tabular}
}

% IMPORTACIÓN DEL TEMPLATE
\input{plantilla_src/template}

% INICIO DE PÁGINAS
\begin{document}

% PORTADA
\templatePortrait

% CONFIGURACIÓN DE PÁGINA Y ENCABEZADOS
\templatePagecfg

% RESUMEN
\begin{abstractd}
Este informe presenta la implementación y análisis de algoritmos de muestreo de Gibbs para dos modelos estocásticos en rejillas: el modelo Hard-Core y el modelo de q-coloraciones. Se desarrollaron implementaciones computacionales eficientes que permiten generar muestras de distribuciones uniformes sobre configuraciones factibles. Los resultados demuestran convergencia efectiva del algoritmo para rejillas de tamaño $K \times K$ con $3 \leq K \leq 20$ y $2 \leq q \leq 10$, caracterizando propiedades estadísticas de las distribuciones estacionarias mediante análisis cuantitativo riguroso.
\end{abstractd}

% TABLA DE CONTENIDOS
\templateIndex

% CONFIGURACIONES FINALES
\templateFinalcfg

% ======================= INICIO DEL DOCUMENTO =======================

\input{plantilla_src/etc/contenido}

% FIN DEL DOCUMENTO
\end{document}
