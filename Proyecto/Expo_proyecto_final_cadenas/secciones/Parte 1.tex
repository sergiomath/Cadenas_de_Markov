\begin{frame}{Introducción}
La ruleta es uno de los juegos de azar más emblemáticos en casinos. A pesar de su aparente simplicidad, constituye un laboratorio matemático para estudiar procesos estocásticos y estrategias de apuesta.

\vspace{0.3cm}

\textbf{Preguntas de investigación:}
\begin{itemize}
    \item ¿Qué estrategia permite al jugador permanecer más tiempo en juego antes de la ruina?
    \item ¿Qué modalidad maximiza la probabilidad de duplicar el capital inicial?
    \item ¿Cómo varía el comportamiento del sistema con el capital inicial?
\end{itemize}

\vspace{0.3cm}

Estas preguntas se abordan mediante un modelo de cadenas de Markov, comparando estrategias (color, par/impar, docena, cuadro, pleno) y analizando la evolución del capital del jugador.

\end{frame}

\begin{frame}{Objetivos}
\begin{block}{Objetivo general}
Analizar el desempeño de diferentes estrategias de apuesta en ruleta mediante simulaciones Monte Carlo de cadenas de Markov, determinando cuáles maximizan la probabilidad de alcanzar una meta de ganancias o minimizan el riesgo de ruina.
\end{block}

\vspace{0.3cm}

\begin{block}{Objetivos específicos}
\scriptsize
\begin{enumerate}
    \item Modelar la evolución del capital mediante cadenas de Markov para distintas modalidades de apuesta.
    \item Calcular probabilidades de transición considerando pagos reales y ventaja de la casa.
    \item Estimar y comparar el tiempo esperado hasta la absorción para cada estrategia.
    \item Establecer cuál estrategia es probabilísticamente más favorable.
\end{enumerate}
\end{block}
\end{frame}

\begin{frame}{Marco teórico}
\begin{block}{Cadenas de Markov}

Una cadena de Markov es un proceso estocástico donde el estado futuro depende únicamente del estado actual, no del pasado (\emph{propiedad markoviana}):

\[
\mathbb{P}(X_{n+1} = j \mid X_n = i, X_{n-1}, \dots, X_0)
= \mathbb{P}(X_{n+1} = j \mid X_n = i) = p_{ij}
\]

Las probabilidades de transición $p_{ij}$ forman la matriz $P = (p_{ij})_{i,j=0}^{N}$, donde cada fila suma 1 ($\sum_{j} p_{ij} = 1$).

\vspace{0.2cm}

\textbf{En nuestro problema:} Los estados son niveles de capital, y las transiciones ocurren según el resultado de cada apuesta.

\end{block}
\end{frame}


\begin{frame}{Marco teórico}
\begin{block}{Ruleta europea}

La ruleta europea tiene 37 casillas (0 a 36), cada una con probabilidad $\frac{1}{37}$.

\vspace{-0.1cm}
\begin{table}[h]
\centering
\scriptsize
\begin{tabular}{lp{4.5cm}cc}
\hline
\textbf{Tipo} & \textbf{Descripción} & \textbf{Pago} & $p$ \\ \hline
Color & Rojo o negro (18 números) & 1:1 & $48.6\%$ \\
Par/Impar & Números pares o impares & 1:1 & $48.6\%$ \\
Docena & Grupo de 12 números (1-12, 13-24, 25-36) & 2:1 & $32.4\%$ \\
Seisena & 6 números en dos filas adyacentes & 5:1 & $16.2\%$ \\
Cuadro & 4 números en esquina & 8:1 & $10.8\%$ \\
Pleno & Un solo número & 35:1 & $2.7\%$ \\ \hline
\end{tabular}
\end{table}

\vspace{-0.1cm}
\textbf{Ventaja de la casa:} El cero garantiza $p < 0.5$ en todas las apuestas.

\end{block}
\end{frame}


\begin{frame}{Marco teórico}
\begin{block}{Modelo de ruina del jugador}

Modelamos el capital como una cadena de Markov con:
\begin{itemize}
    \item \textbf{Capital inicial:} $X_0 = x$
    \item \textbf{Estados absorbentes:} $0$ (bancarrota) y $M$ (meta de ganancia)
    \item \textbf{Apuesta fija:} $b$ unidades en cada ronda
\end{itemize}

El capital evoluciona según:
\[
X_{n+1} =
\begin{cases}
X_n + b, & \text{con probabilidad } p \text{ (gana)}\\
X_n - b, & \text{con probabilidad } q = 1-p \text{ (pierde)}
\end{cases}
\]

En ruleta $p < \frac{1}{2}$, el juego es estructuralmente desfavorable.

\end{block}
\end{frame}





\begin{frame}{Marco teórico}
\begin{block}{Probabilidades de absorción}

¿Cuál es la probabilidad de duplicar el capital antes de arruinarse?

\vspace{0.2cm}

Sea $h_x$ la probabilidad de alcanzar la meta $M$ antes de la ruina $0$, partiendo de capital $x$. La solución depende de si el juego es justo:

\[
h_x =
\begin{cases}
\dfrac{1 - \left( \frac{q}{p} \right)^x}{1 - \left( \frac{q}{p} \right)^M}, & \text{si } p \neq q \text{ (juego sesgado)} \\[10pt]
\dfrac{x}{M}, & \text{si } p = q = \frac{1}{2} \text{ (juego justo)}
\end{cases}
\]

\textbf{En ruleta:} Como $p < q$, entonces $\frac{q}{p} > 1$, por lo que $h_x$ decrece exponencialmente con el capital objetivo. La casa siempre tiene ventaja.

\end{block}
\end{frame}


\begin{frame}{Marco teórico}
\begin{block}{Tiempo esperado de absorción}

¿Cuántas apuestas puede hacer el jugador antes de arruinarse o alcanzar su meta?

\vspace{0.2cm}

El tiempo esperado $t_x$ (número de rondas) satisface una ecuación recursiva:

\[
t_x = 1 + p\, t_{x+b} + q\, t_{x-b}, \quad t_0 = t_M = 0
\]

La solución en forma cerrada para $p \neq q$ es:

\[
t_x = \frac{x}{q-p} - \frac{M}{q-p}
\frac{1 - \left( \frac{q}{p} \right)^x}{1 - \left( \frac{q}{p} \right)^M}
\]

\textbf{Interpretación:} Permite comparar estrategias. Un $t_x$ mayor significa más tiempo de juego (entretenimiento prolongado).

\end{block}
\end{frame}


\begin{frame}{Marco teórico}
\begin{block}{Simulaciones Monte Carlo}

Técnica numérica para estimar cantidades probabilísticas cuando las fórmulas exactas son complejas o intratables.

\vspace{0.2cm}

\textbf{Metodología:}
\begin{enumerate}
    \scriptsize
    \item Simular miles de juegos independientes con las reglas de cada tipo de apuesta
    \item En cada simulación, evolucionar el capital ronda por ronda hasta alcanzar $0$ o $M$
    \item Registrar el resultado (éxito/fracaso) y el número de rondas
\end{enumerate}

\vspace{0.2cm}

\textbf{Estimaciones obtenidas:}
\begin{itemize}
    \scriptsize
    \item \textbf{Probabilidad de éxito:} Proporción de simulaciones que alcanzan $M$
    \item \textbf{Tiempo medio:} Promedio de rondas hasta absorción
\end{itemize}

\vspace{0.1cm}

Con 100,000 simulaciones, los resultados convergen a los valores teóricos (Ley de Grandes Números).

\end{block}
\end{frame}